\documentclass[a4paper,11pt]{article}

\usepackage{a4wide}
\usepackage{microtype}
\usepackage[utf8]{inputenc}
\usepackage[ngerman]{babel}
\usepackage[default,light,semibold]{sourceserifpro}
\usepackage[T1]{fontenc}
\usepackage{mathtools}
\usepackage[dvipsnames]{xcolor}
\usepackage[marginal, norule, perpage]{footmisc}
\usepackage{tabularx}
\usepackage{graphicx}
\usepackage{hyperref}
\usepackage{minted}

%settings
\usemintedstyle{lovelace}
\hypersetup{
  colorlinks=true,
  linkcolor=MidnightBlue,
  urlcolor=MidnightBlue
}

\renewcommand{\thefootnote}{\Roman{footnote}}
\def\arraystretch{1.5}

%custom commands
\newcommand{\lskip}{\vspace{1 em} \\}

%preamble
\title{
    \begin{center}
        \Large{Informationssysteme}\\
        \rule{0.5\textwidth}{0.1 mm}
    \end{center}
    \vspace{1 em}
    \huge{Django \-- Webshop} \vspace{0.5 em} \\
    \large{Protokoll ORM} \vspace{1.5 em}
}

\author{Markus Reichl}

\begin{document}

\maketitle
%\tableofcontents
% \begin{thebibliography}{9}
% \end{thebibliography}

\subsection*{Source}
Die für den Webshop verwendete \texttt{shop/models.py} enthält folgende dokumentierte Zeilen:
\lskip{}
\begin{small}
\begin{minted}{python}
from django.contrib.contenttypes.fields import GenericForeignKey
from django.contrib.contenttypes.models import ContentType
from django.core.exceptions import ValidationError
from django.db import models
from django.utils.datetime_safe import datetime


class Land(models.Model):
    name = models.CharField(max_length=255)
    # Liefert die Länderbezeichnung
    def __str__(self):
        return self.name


class Adresse(models.Model):
    land = models.ForeignKey(Land,on_delete=models.PROTECT)
    strasse = models.CharField(max_length=255)
    hnr = models.PositiveIntegerField()
    plz = models.CharField(max_length=20)
    ort = models.CharField(max_length=255)

    def __str__(self):
        return '{} {}, {} {} - {}'.format(
            self.strasse, self.hnr, self.plz, self.ort, self.land)
    # Mehrere Adressen bekommen einen gemeinsamen Anzeigenamen
    class Meta:
        verbose_name_plural = "Adressen"
\end{minted}

\newpage
\begin{minted}{python}
class Kunde(models.Model):
    name = models.CharField(max_length=255)
    passwort = models.CharField(max_length=255)
    email = models.CharField(max_length=255)
    adresse = models.ForeignKey(Adresse,on_delete=models.CASCADE)
    zuletzt_online = models.DateField(null=True)

    def __str__(self):
        return self.name


class Artikel(models.Model):
    bezeichnung = models.CharField(max_length=255)
    preis = models.DecimalField(max_digits=7, decimal_places=2)
    # Negative Stückzahlen sollen nicht möglich sein
    vstueckz = models.PositiveSmallIntegerField()
    info = models.TextField()

    def clean(self):
        """ Constraints werden in der clean Methode geprüft"""
        if self.preis <= 0:
            raise ValidationError('Preis ist negativ oder 0!')

    def __str__(self):
        return "%s, %s€, %s verfügbar - %s" % \
            (self.bezeichnung, self.preis, self.vstueckz, self.info)


class Feedback(models.Model):
    # Aufgrund der M:N Beziehung mit Kunden ist das ContentType Prinzip notwendig
    content_type = models.ForeignKey(ContentType, on_delete=models.CASCADE)
    object_id = models.PositiveIntegerField()
    content_object = GenericForeignKey()

    kunde = models.ForeignKey(Kunde, on_delete=models.CASCADE)
    artikel = models.ForeignKey(Artikel, on_delete=models.CASCADE)
    datum = models.DateTimeField()
    bewertung = models.TextField()

    def __str__(self):
        return "%s on %s - %s | %s" % \
            (self.kunde, self.artikel.bezeichnung, self.datum, self.bewertung)
\end{minted}

\newpage
\begin{minted}{python}
class Lieferbarkeit(models.Model):
    land = models.ForeignKey(Land, on_delete=models.CASCADE)
    artikel = models.ForeignKey(Artikel, on_delete=models.CASCADE)
    mwst = models.DecimalField(max_digits=2, decimal_places=2)

    def clean(self):
        if self.mwst <= 0:
            raise ValidationError('Mehrwertsteuer ist negativ oder 0!')

    def __str__(self):
        return "%s, %s, %s" % \
            (self.artikel.bezeichnung, self.land, self.mwst)


class Bestellung(models.Model):
    status_types = (('b', 'bestellt'), ('v', 'versendet'), ('s', 'storniert'))

    artikel = models.ManyToManyField(Artikel)
    kunde = models.ForeignKey(Kunde, on_delete=models.CASCADE)
    datum = models.DateTimeField()
    # Eine Auswahl an Werten kann durch den Parameter "choices" vorgegeben werden.
    # Dieser übernimmt ein Iterable aus Tupeln, hier "status_types".
    status = models.CharField(
        max_length=255, choices=status_types, default="bestellt")
    # Da "Adresse" mehrmals als "ForeignKey" agiert muss bei der Benutzung ein
    # "related_name" festgelegt werden, um diese zu unterscheiden.
    lieferadresse = models.ForeignKey(
        Adresse, on_delete=models.CASCADE, related_name="lads", null=False)
    rechnungsadresse = models.ForeignKey(
        Adresse, on_delete=models.CASCADE, related_name="rads", null=False)

    def clean(self):
        # Django's datetime.now() ist mit dem SQL Datentypen DATE kompatibel
        if self.datum <= datetime.now():
            raise ValidationError('Datum liegt in der Vergangenheit!')

    def __str__(self):
        return "%s, %s, %s, %s, %s, %s" % \
               (self.artikel, self.kunde, self.datum,
                self.status, self.lieferadresse, self.rechnungsadresse)


class Bestellartikel(models.Model):
    artikel = models.ForeignKey(Artikel, on_delete=models.CASCADE)
    bestellung = models.ForeignKey(Bestellung, on_delete=models.CASCADE)
    anzahl = models.PositiveSmallIntegerField()

    def __str__(self):
        return "%s, %s, %s" % (self.artikel, self.bestellung, self.anzahl)

\end{minted}

\newpage
\begin{minted}{python}
# Die folgenden Modelle erben von Artikel und benätigen daher auch seine Referenz
class Bluray(Artikel):
    regisseur = models.CharField(max_length=255)
    jahr = models.PositiveSmallIntegerField()
    genre = models.CharField(max_length=255)
    dauer = models.IntegerField()

    def __str__(self):
        return "%s, %s, %s, %s, %s" % \
            (self.bezeichnung, self.regisseur, self.jahr, self.dauer, self.genre)
class Buch(Artikel):
    autor = models.CharField(max_length=255)
    verlag = models.CharField(max_length=255)
    seiten = models.PositiveSmallIntegerField()
    isbn = models.CharField(max_length=255)

    def __str__(self):
        return "%s, %s, %s, %s, %s" % \
        (self.bezeichnung, self.autor, self.verlag, self.isbn, self.seiten)


class SonstigerArtikel(Artikel):
    pass
\end{minted}
\end{small}

\subsection*{Weiteres}
Wichtige Stellen wurden bei ihrem ersten vorkommen dokumentiert, weitere Entscheidungen werden in folgendem Absatz erläutert.

\subsubsection*{Constraints}
Die SQL Constraints sollten bereits vor der Speicherung in die Datenbank geprüft werden.
In Django übernimmt diese Aufgabe die Methode \texttt{models.Model.clean}, welche vom Entwickler überschrieben wird.
Innerhalb dieser werden dann \texttt{ValidationError}s geworfen.

\subsubsection*{Vererbung}
In Django sind Paradigmen wie Vererbung und Polymorphie generell möglich,
nicht anwendbar ist letzteres jedoch, wenn es sich bei der Stammklasse um eine abstrakte Klasse handelt.

\subsubsection*{ManyToMany}
In Django wird die M:N Beziehung standardmäßig gesetzt. Wie man diese Einstellungen manuell vornimmt wird in https://docs.djangoproject.com/en/1.11/ref/models/fields/ erläutert.

\end{document}
